\section{Al/Mg/Al复合板材力学性能分析}

	\subsection{拉伸性能分析}
	不同变形量轧制后Al/Mg/Al复合板材室温拉伸性能如表\ref{tab:butongbianxingliang}所示。由表可知,铝镁板材45$\%$轧制复合后,抗拉强度为187.390MPa,屈服强度为68.97MPa,延伸率为43.13$\%$。实验所用原始AZ31B镁合金的抗拉强度为199.79MPa,屈服强度为95.47MPa,延伸率为15.3$\%$;1060纯铝的抗拉强度为106.76MPa,屈服强度为73.06MPa,延伸率为25.75$\%$。\par
		
		
	%顶部对齐,将空白集中到页面底部
	%\raggedbottom
		
		\vspace{-0.5em}
		\begin{table}[H]
			\centering
			\caption{\ \ 不同变形量轧制后Al/Mg/Al复合板材室温拉伸性能}
			\zihao{5}
			\label{tab:butongbianxingliang}
			\begin{tabularx}{0.8\linewidth}{*{4}{>{\centering\arraybackslash}X}}
				\toprule[1.5pt]
				变形量	&抗拉强度/MPa	&屈服强度/MPa	&延伸率$\%$   \\ 
				\midrule[0.5pt]
				AZ31B镁合金	&199.79	&95.47	&15.3  \\ 
				1060纯铝	&106.76	&73.06	&25.75  \\ 
				45$\%$	&187.39	&68.97	&43.13  \\ 
				62$\%$	&187.38	&161.49	&30.7  \\ 
				73$\%$	&230.88	&195.30	&10.87  \\ 
				83$\%$	&198.49	&188.02	&7.47  \\ 
				\bottomrule[1.5pt]
			\end{tabularx}
		\vspace{-1em}
		\end{table}
	\subsection{剪切试验分析}
		\subsubsection{轧制变形率对复合材料剪切强度的影响}
		随着压下量的提高,结合强度逐渐提高,达到峰值后结合强度下降。峰值出现在压下量为73$\%$条件下,剪切强度为120.45MPa。轧制压下率只有45$\%$时,两种材料的结合强度较低。\par
		\subsubsection{退火温度对复合材料结合强度的影响}
		对各曲线进行线性拟合曲线,拟合后得到各曲线的斜率的平均值为0.345,截距的平均值为72.8。得到拟合方程为 。\par
	\subsection{硬度实验分析}
		
		\subsubsection{轧制变形率对复合材料硬度的影响}
	  
		复合轧制前,Z31B镁合金的平均硬度为54.52HV1,显微硬度较低。经过一道次轧制后45$\%$压下后,可以看到硬度明显提高,平均硬度达到71.62HV1,提高了31.36$\%$。\par
	  	\subsubsection{退火温度对复合材料硬度的影响}
	  
		与轧制态板材相比,退火处理会使板材硬度下降。200℃-250℃退火2小时所得复合板材硬度高于原始母材,对比各温度梯度退火条件下材料的硬度值,硬度值下降较少。\par
		\subsubsection{冲击实验分析}
		
		因为本实验使用的冲击试样并非标准试样,不同变形量下板材厚度也不同,测得的吸收能量值不同,实验所得的冲击吸收功不存在换算关系,也无法对比,所以此处不做定量分析。轧制力模型计算值与实测值比较如图\ref{fig:zhazhilijisuan}所示。\par
		\begin{figure}[H]
			\vspace{-1em}
			\centering
			\includegraphics[width=0.8\textwidth]{images/zhazhilijisuan}
			\caption{轧制力计算值与实测值比较}
			\label{fig:zhazhilijisuan}
			\vspace{-1.5em}
		\end{figure}
		\subsubsection{跨页长表格}
		
		因为本实验使用的冲击试样并非标准试样,不同变形量下板材厚度也不同,测得的吸收能量值不同,实验所得的冲击吸收功不存在换算关系,也无法对比,表\ref{tab:lizishuju}所以此处不做定量分析。\par
		
		\vspace{-0.5em}
		\begin{small}
			\begin{longtabu} to 0.8\linewidth{*{6}{>{\centering\arraybackslash}X}}	
				\caption{例子数据}\label{tab:lizishuju}  \\
				
				\toprule[1.5pt]
				主题1  & 主题2  & 主题3  & 主题4  & 主题5  & 主题6    \\ 
				\midrule[0.5pt]
				\endfirsthead
				%\multicolumn{6}{c}{续表~\thetable\hskip1em 例子数据}\\
				\multicolumn{6}{r}{表~\thetable\hskip0.5em(续表)}\\
				\toprule[1.5pt]
				主题1  & 主题2  & 主题3  & 主题4  & 主题5  & 主题6    \\ 
				\midrule[0.5pt]
				\endhead
				\bottomrule[1.5pt]
				\endfoot
				\endlastfoot
				外观	&屏幕	&效果	&管理	&降价	&使用者     \\ 
				速度	&买	&差	&骂人	&京东	&google   \\ 
				材质	&不	&一般	&见得	&刚买	&元顶   \\ 
				运行	&是	&换	&牛批	&价格	&翻来覆去   \\ 
				充电	&没有	&保价	&乌合之众	&降	&较   \\ 
				轻薄	&还	&质量	&一群	&200	&目的地   \\ 
				挺	&用	&过	&google	&申请	&不合理   \\ 
				特色	&问题	&垃圾	&不合理	&一个月	&舒适感   \\ 
				程度	&没	&有点	&标志	&退	&出毛病   \\ 
				打开	&说	&月	&跳出	&电	&衡水   \\ 
				
				\bottomrule[1.5pt]
				
			\end{longtabu}
			\vspace{-1.5em}
		\end{small}
		因为本实验使用的冲击试样并非标准试样,不同变形量下板材厚度也不同,测得的吸收能量值不同,实验所得的冲击吸收功不存在换算关系,也无法对比,所以此处不做定量分析。\par
		
		\subsubsection{代码块}
		
		因为本实验使用的冲击试样并非标准试样,不同变形量下板材厚度也不同,也无法对比,所以此处不做定量分析。这里以\Cref{code_api_service} 为例,代码块使用示例:
		\vspace{-0.8em}
		\begin{lstlisting}[caption=APIService , label=code_api_service]
/** 
* 登录 
*/
@POST("user/login")
Call<APIResponse<LoginResponse>> login(@Body LoginRequest request);

/** 
* 获取单个用户详细信息
*/
@GET("user/{imid}/info")
Call<APIResponse<UserInfo>> getUserInfo(@Path("imid") String imId);
		\end{lstlisting}
		\vspace{-0.5em}
	    因为本实验使用的冲击试样并非标准试样,不同变形量下板材厚度也不同,也无法对比,所以此处不做定量分析。\par   
 \clearpage