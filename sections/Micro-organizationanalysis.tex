\section{Al/Mg/Al复合板材微观组织分析}

	为探究不同轧制条件下镁铝复合板材的性能变化,本章通过分析金相组织、扫描电镜分析、XRD衍射分析、拉伸实验、弯曲实验、冲击实验和剪切实验及表面维氏硬度测试的实验结果,探究镁铝复合板材的组织、性能变化的特点并探究其原因。\par
	
	\subsection{金相组织分析}		  
		\subsubsection{压下量变化对金相组织的影响}
		\begin{enumerate}
			\item 镁层ND-RD面金相组织变化
			
			不同压下量下复合板材中间镁层金相显微组织如图3.1所示。其中辊面温度为300℃,轧制速度为0.03 m/s,RD为轧制方向,ND为厚度方向。
			
			\item 结合面金相组织变化
			
			如图3.2所示为辊面温度为300℃,不同压下率下轧后Al/Mg/Al复合板镁层与铝层结合面金相组织。
		\end{enumerate}
		\subsubsection{退火温度对金相组织的影响}
		在本次实验中,经过轧制后的镁铝复合板材,因为镁合金的滑移系较少,在轧制过程中不易发生宏观屈服,因此会在晶界处造成应力集中,Al/Mg/Al层合板中的外层铝层所承受的较大变形是由板与辊之间的摩擦引起的相对较大的剪切应变引起的,导致了铝层很强的加工硬化。此时镁铝复合板的内能、强度硬度都比较高,处于亚稳态状态。\par

		\subsubsection{晶粒大小与平均尺寸分布图}
		对比原始板材的晶粒大小分布可发现,原始板材本身的晶粒尺寸分布很不均匀,在各个轧制压下率下,而在压下率提升后,均匀性有了很大程度的改善,其中,73$\%$和83$\%$压下量板材镁层晶粒尺寸最细小,83$\%$压下量板材镁层晶粒均匀性最高。分布的均匀性也影响到了复合板材的力学性能。\par
	\subsection{扫描分析}			
		\subsubsection{Al/Mg/Al复合板材微观形貌}
		本实验中镁层和铝层之间的结合良好,界面之间没有裂缝、空洞或界限。从图中可以看到随着变形率的提高,镁层的厚度逐渐降低,两种组分金属在各个变形量轧制下的变形基本均匀并保持了良好的连续性,增大压下率,当变形量达到73$\%$,可以观察到波纹结构。\par
		\subsubsection{Al/Mg/Al复合板材剥离面微观形貌}
		\begin{enumerate}
			\item 镁层剥离面微观形貌
			
			通过扫描数据可知,镁层上生成的中间相的原子镁元素的质量百分比为50.74$\%$,铝元素的质量百分比为46.69$\%$,原子比接近17:12。
			
			\item 铝层剥离面微观形貌
			
			扫描数据显示,镁层上生成的中间相的原子镁元素的质量百分比为44.97$\%$,铝元素的质量百分比为53.18$\%$,原子比接近2:3。
		\end{enumerate}
	\subsection{XRD分析}	
		为了进一步确认中间结合层物相的成分并探究不同退火条件对中间层含量的影响,对各组样品进行了XRD实验,利用JADE软件对实验数据进行分析。\par
		对数据进行分析和拟合得到元素扩散模型见公式\eqref{e1}和\eqref{e2}:\par
	\vspace{-2em}
	\begin{equation} \label{e1}
		\begin{split}
			&f(x,y)=[f(1,0)-f(0,0)]x+[f(0,1)-f(0,0)]y \\
			&+[f(1,1)+f(0,0)-f(0,1)]xy+f(0,0) 
		\end{split}
	\end{equation}
	\vspace{-1em}
	\begin{equation} \label{e2}
		\begin{split}
			f(x,y)&=[f(1,0)-f(0,0)]x+[f(0,1)-f(0,0)]y \\
			&=[f(1,1)+f(0,0)-f(0,1)]xy+f(0,0) 
		\end{split}
	\end{equation}	
		
 \clearpage